%!TEX program = xelatex
\documentclass[a4paper]{article}

\usepackage{ctex}
\usepackage{fontspec, xunicode, xltxtra}
\usepackage{fancyhdr}
\pagestyle{fancy}
\usepackage{amsmath,amssymb}
\usepackage{lastpage}{}
\usepackage{inden tfirst}
\usepackage{listings}
\usepackage{xcolor}
\usepackage{listings}
%\usepackage{graphicx}
\usepackage{courier}
\lstset{
         basicstyle=\footnotesize\ttfamily, % Standardschrift
         numbers=left,               % Ort der Zeilennummern
         numberstyle=\tiny,          % Stil der Zeilennummern
         %stepnumber=5,               % Abstand zwischen den Zeilennummern
         numbersep=5pt,              % Abstand der Nummern zum Text
         tabsize=4,                  % Groesse von Tabs
         extendedchars=true,         %
         breaklines=true,            % Zeilen werden Umgebrochen
         keywordstyle=\color{blue!70}\bfseries,
         commentstyle=\rmfamily\itshape,
         stringstyle=\color{white}\ttfamily, % Farbe der String
         showspaces=false,           % Leerzeichen anzeigen ?
         showtabs=false,             % Tabs anzeigen ?
         frame=b,
         xleftmargin=17pt,
         framexleftmargin=17pt,
         framexrightmargin=5pt,
         framexbottommargin=4pt,
         %backgroundcolor=\color{lightgray},
         showstringspaces=false      % Leerzeichen in Strings anzeigen ?
}

\usepackage{caption}
\DeclareCaptionFont{white}{\color{white}}
\DeclareCaptionFormat{listing}{\colorbox[cmyk]{0.43, 0.35, 0.35,0.01}{\parbox{\textwidth}{#1#2#3}}}
\captionsetup[lstlisting]{format=listing,justification=raggedright,labelfont=white,textfont=white, singlelinecheck=false, margin=0pt, font={sf,bf,footnotesize}}

\renewcommand{\today}{\number\year /\number\month /\number\day}

\setmainfont{Hiragino Sans GB}
\lhead{1500012775 钱昊}
\rhead{Page \thepage{} of \pageref{LastPage}}
\begin{document}
\section*{1}
存储数据大小为$n$
\begin{lstlisting}[language=C++]
int top(){
	int res;
	while(!A.empty()){
	     int tmp =  A.front();
	     A.pop_front();
         B.push_back(tmp);
         if(A.empty())
             res = tmp;	
	}
	while(!B.empty()){
	     A.push_back(B.front());
	     B.pop_front();
	}
	return res;
}
\end{lstlisting}
时间复杂度O(n)\\
\\
\begin{lstlisting}[language=C++]
void pop(){
	while(!A.empty()){
	int tmp = A.front();
	A.pop_front();
	if(!A.empty())
        B.push_back(tmp);
	}
	while(!B.empty()){
	    A.push_back(B.front());
	    B.pop_front();
	}
}
\end{lstlisting}
时间复杂度O(n)\\
\\
\begin{lstlisting}[language=C++]
void push(int &x){
	A.push_back(x);
}
\end{lstlisting}
时间复杂度O(1)\\
\\
\begin{lstlisting}[language=C++]
bool empty(){
	return A.empty();
}
\end{lstlisting}
时间复杂度O(1)\\
\\
\section*{2}
\subsection*{formula}
出栈顺序等价于求n个0,n个1组成的排列个数。\\
该排列满足性质,对于任意其任意子排列[0,a),该子排列满足0的个数大于等于1\\
总的排列数$C_{2n}^n$\\
求不满足要求的排列数\\
对于任何不满足的排列,存在一个子排列,使得子排列有$x+1$个1,$x$个0,然后将改子排列的0,1互换,就变成了n+1个0,n-1个1的排列,且这种转换一一对应。个数为$C_{2n}^{n+1}$\\
Catalan数=$C_{2n}^n-C_{2n}^{n+1}=\frac{1}{n}C_{2n}^n$
\subsection*{Prove}
这道题等价于求当$i<j<k$,所有可能出现的$P_i,P_j,P_k$的顺序\\
1. j 入栈时, i 已经出栈\\
1.1 j出栈后, k入栈, 出站顺序i,j,k $P_i<P_j<P_k$\\
1.2 k先入栈,出站顺序i,k,j$P_i<P_k<P_j$\\
2 j 入栈时, i未出栈, 同时先于k入栈出栈 \\
2.1 k先于i出栈 出栈顺序j,k,i  $P_k<P_i<P_j$\\
2.2 k后于i出栈 出栈顺序j,i,k  $P_j<P_i<P_k$ \\
3 k入栈,i,j都没出栈,出栈顺序k,j,i $P_k<P_j<P_i$\\
综上,对于$P_i,P_j,P_k$的排列,只有$P_j<P_k<P_i$没有出现\\
所以不存在下表$i,j,k$,使得$P_j<P_k<P_i$\\
\end{document}
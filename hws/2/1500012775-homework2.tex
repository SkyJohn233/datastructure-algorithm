%!TEX program = xelatex
\documentclass[a4paper]{article}

\usepackage{ctex}
\usepackage{fontspec, xunicode, xltxtra}
\usepackage{fancyhdr}
\pagestyle{fancy}
\usepackage{amsmath,amssymb}
\usepackage{lastpage}{}
\usepackage{inden tfirst}
\usepackage{listings}
\usepackage{xcolor}
\usepackage{listings}
%\usepackage{graphicx}

\setmainfont{Hiragino Sans GB}
\lhead{1500012775 钱昊}
\rhead{Page \thepage{} of \pageref{LastPage}}
\begin{document}
\section*{1}
\begin{lstlisting}[language=C++]
bool isCircle(Listnode *head){
	Listnode *slow=head,*fast=head;
	while(fast){
		fast=fast->next;
		if(fast)
		fast=fast->next;
		slow=slow->next;
		if(slow==fast)
			break;
	}
	return fast==nullptr?false:true;
}

\end{lstlisting}
空间复杂度:\\
O(1)\\
时间复杂度\\
假如没有环 时间复杂度为$O(n)$;假如有环,令$K$为从head开始到环开始地方的长度,$L$为环的长度,令$D=K\mod L$,时间复杂度为$O(K+(L-D))$。总结来说,时间复杂度$O(n)$

\section*{2}
\subsection*{2.1}
设计:\\
跳表共有n层单链表组成。\\
level1 层单链表包含了原单链表的所有元素\\
level2 层单链表包含了$\frac{n}{2}$元素,相邻元素的间隔为2\\
level3 层单链表包含了$\frac{n}{4}$元素,相邻元素间隔为4\\
以此类推\\
n的大小由原单链表的长度L决定\\
第二步\\
将每层的链表连起来\\
具体操作:\\
\begin{lstlisting}[language=C++]
struct Skiplistnode{
	int val;
	Skiplistnode *next,*down;
}
\end{lstlisting}
其中next指针指向同level链表的下一个元素,down指针指向level-1层链表的相同元素。(这个由构造保证一定存在)

这里,我们首先需要将原链表进行排序。
查找第i个元素的伪代码
\begin{lstlisting}
fuction search(Skiplistnode *cur,int tar){
    if cur->val == tar
        return 
	if cur->next == nullptr || cur->next->val<=tar
	   search(cur->next,tar)
	search(cur->down,tar)
}

\end{lstlisting}
\subsection*{2.2}
时间复杂度分析:\\
由于SkipList的构造可知,通过判断cur->next->val 与 tar的大小,将整个search的范围缩小了一半\\
可以得到递推方程\\
T(n)=T(n/2)+O(1)\\
得到T(n)=O(logn)
\end{document}